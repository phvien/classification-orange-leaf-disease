% \begin{Huge}
% {\raggedleft \textbf{Tóm tắt}}
% \end{Huge}\vspace{1.6cm}
\chapter*{\Huge Tóm tắt}
Trong luận văn này, chúng tôi đề xuất đề tài ứng dụng máy học véc-tơ hỗ trợ (Support Vector Machines - SVM) trong phân loại bệnh trên lá cam sử dụng các đặc trưng SIFT (Scale-Invariant Feature Transform) kết hợp mô hình túi đựng từ trực quan (Bag of Visual Words - BoVW), Color, HOG (Histogram of Oriented Gradient), GIST và ResNet (Residual Networks). Bước rút trích đặc trưng từ ảnh lá bệnh, được thu thập từ 2 tỉnh Đồng Tháp và Hậu Giang. Chúng tôi đề xuất huấn luyện mô hình máy học SVM sử dụng các đặc trưng SIFT, Color, HOG, GIST và ResNet để thực hiện rút trích dựa trên các đặc trưng khác nhau của từng loại bệnh. Kết quả thực nghiệm trên tập dữ liệu gồm 711 ảnh của 5 loại, trong đó 4 loại là bệnh và 1 loại lá khỏe thường được xuất hiện phổ biến trong các nhà vườn cho thấy máy học SVM sử dụng các đặc trưng SIFT, Color, HOG, GIST và ResNet đạt đến 90.84\% độ chính xác trên tập kiểm tra, cao hơn so với mô hình K láng giềng (K-Nearest Neighbors - KNN) có độ chính xác chỉ 42.95\%.\\

\noindent \textbf{Từ khóa}--\emph{Phân loại ảnh, phân loại bệnh trên lá cam, các phương pháp trích chọn đặc trưng, SIFT, BoVW, Color, HOG, GIST, ResNet, máy học véc-tơ hỗ trợ (SVM)}